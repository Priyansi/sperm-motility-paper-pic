\documentclass[11pt]{article}

\usepackage{times}
\usepackage{geometry}
\usepackage{authblk}

\geometry{left=1in, right=1in, top=0.2in, bottom=0.7in}
\setlength{\parindent}{0pt}
\renewcommand{\refname}{\centering \normalfont \scshape References}

%opening
\title{\huge \textbf{Predicting Semen Motility using three-dimensional Convolutional Neural Networks}}

\author[1]{Priyansi}
\author[2]{Biswaroop Bhattacharjee}
\author[3]{Junaid Rahim}
\affil[1]{School of Computer Engineering, KIIT, 1905110@kiit.ac.in}
\affil[2,3]{School of Computer Engineering, KIIT}

\date{}



\begin{document}

\maketitle

\begin{abstract}

Manual and computer aided methods to perform semen analysis is time consuming, requires extensive training and prone to human error. The use of classical machine learning and deep learning based methods using videos to perform semen analysis have yielded good results. The state of the art method uses regular convolutional neural networks to perform quality assessments on the provided sample. In this paper we propose an improved deep learning based approach using three-dimensional convolutional neural networks to process microscopic semen videos.
	
\end{abstract}

{\bf Index Terms - } Semen Motility Prediction, Deep Learning, Residual Networks, 3D Convolutional Neural Networks

\section{\normalfont \textsc{Introduction}}

Mention\cite{kour2014real}

\section{\normalfont \textsc{Basic Concepts/Technology Used}}
\section{\normalfont \textsc{Literature Review}}
\section{\normalfont \textsc{Proposed Model}}
\section{\normalfont \textsc{Implementation and Results}}
\section{\normalfont \textsc{Conclusion}}

\bibliographystyle{unsrt}  
\bibliography{references}

\end{document}

